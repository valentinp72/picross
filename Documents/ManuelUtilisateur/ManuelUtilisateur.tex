\documentclass[a4paper, 12pt]{report}


%%%%%%%%%%%%%%%%%%
%   Liste des packages utilisés  %
%%%%%%%%%%%%%%%%%%

\usepackage{float}
\usepackage{lmodern} %Pack de police
\usepackage{graphicx}
\usepackage[utf8x]{inputenc}
\usepackage[T1]{fontenc}
\usepackage[francais]{babel}
\usepackage{caption}
\usepackage[top=2cm, bottom=2cm,left=2cm, right=2cm]{geometry}
\usepackage{setspace}
\usepackage{fancyhdr}
\usepackage{titlesec, blindtext, color} % titres spéciaux + couleur pour les chapter

% on transforme les chapters en juste le numéro suivi du titre, avec un barre grisse
\definecolor{gray75}{gray}{0.75}
\newcommand{\hsp}{\hspace{20pt}}
\titleformat{\chapter}[hang]{\Huge\bfseries}{\thechapter\hsp\textcolor{gray75}{|}\hsp}{0pt}{\Huge\bfseries}


%Définition du style des bords de page
\pagestyle{fancy}
\renewcommand{\chaptermark}[1]{\markboth{\bsc{\thechapter{}- } #1}{}}
\lhead{}
\chead{}
\rhead{\leftmark}
\lfoot{Groupe n\up{o}2}
\cfoot{}
\rfoot{Page \thepage}

\fancypagestyle{plain}{%
    \lhead{}
    \chead{}
    \rhead{}
    \renewcommand{\headrulewidth}{0pt}
    \lfoot{Groupe n\up{o}2}
    \cfoot{}
    \rfoot{Page \thepage}
}

\begin{document}

%%%%%%%%%%%
%  Page de garde  %
%%%%%%%%%%%
\begin{titlepage}



	\begin{spacing}{1.5}
			\begin{minipage}{0.4\textwidth}
					\includegraphics[width=3cm]{logo.png}
			\end{minipage}
			\begin{minipage}{0.5\textwidth}\raggedleft
					RuÞycross\\
			\end{minipage}
						\vspace*{\fill}

	\end{spacing}



	\begin{center}
		\begin{spacing}{2}
		    \hrule \vspace{1cm}
			\textbf{\huge Manuel Utilisateur}\\
			\vspace{1cm}
			\begin{minipage}{0.4\textwidth}
					\centerline{\includegraphics[width=3cm]{test_logo_2.png}}
			\end{minipage}
			\vspace{1cm}
			\hrule

			\vspace*{\fill}


		\end{spacing}

		\begin{spacing}{1.15}
			\large\textbf{Groupe n\up{o}2} :\\
			\large
			\textsc{Brinon} Baptiste\\
			\textsc{Brocherieux} Thibault\\
			\textsc{Cohen} Mehdi\\
			\textsc{Debonne} Valentin\\
			\textsc{Lardy} Anthony\\
			\textsc{Mottier} Emeric\\
			\textsc{Pastouret} Gilles\\
			\textsc{Pelloin} Valentin\\
			\vspace*{\fill}
			%\textbf{Groupe n\up{o}2} \\
			\textnormal{\large Licence Informatique\\ Le Mans Université\\ \today}
		\end{spacing}

	\end{center}
\end{titlepage}

%%%%%%%%%%
%    Sommaire    %
%%%%%%%%%%
\renewcommand{\contentsname}{Sommaire}
\tableofcontents
\thispagestyle{empty}
\thispagestyle{plain}



\chapter{Pré-requis pour l'installation}
\thispagestyle{empty}
\thispagestyle{plain}

    Pour pouvoir exécuter correctement l'application, vous devez au préalable :
    \begin{itemize}
        \item Posséder un ordinateur sous MacOS ou Linux;
        \item Avoir installé une version de Ruby ultérieure à la 2.2.2;
        \item Avoir installé une version de la librairie Gtk ultérieure à la 3.22;
    \end{itemize}


\chapter{Règles du jeu}
\thispagestyle{empty}
\thispagestyle{plain}


		\section{Le but du jeu}

            Le but du Picross est de noircir les cases d'une grille de jeu, afin de faire apparaître une image, un dessin... Ceci à l'aide des indices se situant en marge de la grille.

		\section{Les cases à noircir}

            Une grille comporte des indices en sa marge. Une ligne ou une colonne peut posséder un certain nombre d'indices :
            \begin{itemize}
                \item Les indices se situant à gauche de la grille permettant de connaître le nombre de cases à noircir sur la ligne correspondante.;
                \item Les indices se situant en haut de la grille permettent de connaître le nombre de cases à noircir sur la colonne correspondante.;
            \end{itemize}

	        Ainsi, une grille possédant un nombre 5 devant une ligne ou une colonne indique qu'il y a 5 cases consécutives à noircir.

	        Une grille possédant la séquence de nombre 2 3 devant une ligne ou une colonne indique qu'il y a un block de 2 cases à noircir, suivis d'au moins une case vide, puis un block de 3 cases à noircir.


		\section{Les cases faciles à noircir}

            Certaines astuces permettent de repérer les cases faciles à noircir, en voici quelques unes :
            \begin{itemize}
                \item Prenons par exemple un indice valant 10 sur une ligne d'une grille 10x10. Cela signifie que toutes les cases de la ligne en question sont à noircir.;
                \item Prenons maintenant l'exemple d'une ligne ou une colonne dont l'indice est 3, et la première case est déjà noircie. Cela signifie que les cases à noircir sont obligatoirement les 3 premières.;
                \item Un dernier exemple : nous avons une grille 10x10, l'indice d'une des ligne vaut 7. On peut dans ce cas noircir les 4 cases centrales, ces cases seront noircies quelque soit la solution de cette ligne. Cette astuce fonctionne dès qu'une ligne ou une colonne possède un indice unique, supérieur à la moitié du nombre de cases de la ligne ou de la colonne en question.;
            \end{itemize}


		\section{Les cases à éliminer}

            Certaines cases sont à éliminer, elles ne sont donc pas à noircir, cela permet ainsi de voir plus clair dans la résolution d'une grille.

            Par exemple, les cases restantes d'une ligne contenant 5 cases à noircir, que vous avez déjà trouvées, sont à éliminer.

		\section{}


		\section{}



\chapter{Guide pas à pas}
\thispagestyle{empty}
\thispagestyle{plain}

	\section{}


	\section{}





	\section{}


	\section{}


	\section{}



	\section{}


	\section{}




%	``I always thought something was fundamentally wrong with the universe'' \citep{adams1995hitchhiker}
		\bibliographystyle{plain}
		\bibliography{references}

\end{document}
