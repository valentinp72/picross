\documentclass{report}


%%%%%%%%%%%%%%%%%%
%   Liste des packages utilisés  %
%%%%%%%%%%%%%%%%%%

% (oui y'en a 95% qui sont inutiles ^^)

\usepackage{amssymb}
\usepackage{array}
\usepackage{hyperref}
\usepackage{booktabs}
\usepackage{multirow}
\usepackage{float}
\usepackage{lmodern} %Pack de police
\usepackage{color}
\usepackage[dvipsnames]{xcolor}
\usepackage{graphicx}
\usepackage[utf8x]{inputenc}
\usepackage[T1]{fontenc}
\usepackage{natbib}
\usepackage[francais]{babel}
\usepackage{caption}
\usepackage{listings}
\usepackage{booktabs}
\usepackage[top=2cm, bottom=2cm,left=2cm, right=2cm]{geometry}
\usepackage{blindtext}
\usepackage{setspace}
\usepackage{graphicx}
\usepackage{titlesec, blindtext, color} % titres spéciaux + couleur pour les chapter

% on transforme les chapters en juste le numéro suivi du titre, avec un barre grisse
\definecolor{gray75}{gray}{0.75}
\newcommand{\hsp}{\hspace{20pt}}
\titleformat{\chapter}[hang]{\Huge\bfseries}{\thechapter\hsp\textcolor{gray75}{|}\hsp}{0pt}{\Huge\bfseries}

\begin{document}


%%%%%%%%%%%
%  Page de garde  %
%%%%%%%%%%%
\begin{titlepage}
	\begin{center}
	
		\begin{spacing}{1.5}
			Projet Picross\\
			\vspace*{\fill}
		\end{spacing}
		
		\begin{spacing}{2.5}
			\textbf{\Huge Application de création et d'aide à la résolution de puzzle \textit{picross}}\\[0.5cm]
			\textbf{\huge Cahier d'analyse et conception} \\
			\vspace*{\fill}
			\textit{Étudiants :}
		\end{spacing}

		\begin{spacing}{1.15}
			\large
			\textsc{Brinon} Baptiste\\
			\textsc{Brocherieux} Thibault\\
			\textsc{Cohen} Mehdi\\
			\textsc{Debonne} Valentin\\
			\textsc{Lardy} Anthony\\
			\textsc{Mottier} Emeric\\
			\textsc{Pastouret} Gilles\\
			\textsc{Pelloin} Valentin\\
			\vspace*{\fill}
			\textbf{Groupe n°2} \\
			\textnormal{\large Licence Informatique\\ Le Mans Université\\ \today}
		\end{spacing}
		
	\end{center}
\end{titlepage}


%%%%%%%%%%
%    Sommaire    %
%%%%%%%%%%
\renewcommand{\contentsname}{Sommaire}
\tableofcontents


\chapter{Présentation}

	\section{Introduction}

		Dans le cadre de l'unité d'enseignement "Génie logiciel 2" de la Licence d'informatique de Le Mans Université, les étudiants de troisième année sont amenés à travailler sur un projet de développement d'une application. \\
		Ce document décrit ...

	
 	\section{Objectif de l'application}		
		Nous devons réaliser un jeu de type picross (aussi appelé nonogramme, logigramme ou hanjie) permettant à un utilisateur de résoudre des grilles et de l'aider dans sa réalisation.
		
		\section{Outils}
		
		// Language logiciel ...
		
\chapter{Conception générale}

    \section{conduite de projet}
      // diagramme de Gant
      
    \section{Fonctionnalité}
      // Cas d'utilisation + descriptif
      // diagrammes de séquences + descriptif 
    
		\section{Image}
			// Schéma liens entre écrans
			// Description
			
\chapter{Conception détaillé}

    \section{ ,, }
      // diagramme de classe + descriptif

    
\chapter{Annexes}

      // maquettes

\end{document}
